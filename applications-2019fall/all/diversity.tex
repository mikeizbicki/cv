\documentclass[12pt]{article}

\usepackage[margin=1in]{geometry}
\usepackage{url}

\usepackage{titlesec}
%\titleformat{\section}[runin]
    %{\normalfont\bfseries}{\thesection}{1em}{}
\titleformat{\subsection}[runin]
    {\normalfont\bfseries}{\thesubsection}{0em}{\vspace{-0.00in}}

%\setlength{\parskip}{15pt plus4mm minus3mm}

\newcommand{\newpar}{\vspace{0.05in}\noindent}
\newcommand{\ignore}[1]{}

\begin{document}


%\section*{Student Success Statement for Mike Izbicki}
\section*{Diversity Statement for Mike Izbicki}

I want to live in a world where everyone feels included and supported.
I've worked to create such a world by volunteering in local schools, 
by fostering global collaborations, 
and by helping veterans.
I strive to translate this work into my classrooms so that all students are able to perform their best.

\vspace{-0.15in}
\subsection*{Locally.}

One reason for the lack of diversity in computing is that many K12 students grow up believing computing is ``not for them.''
For example, students from poorer neighborhoods may not know anyone involved in computing,
or even anyone who went to college.
I want to help change this.
So I volunteered for a summer program to teach programming skills to elementary and middle school students called Code Avengers.
This program was run through a school in Moreno Valley,
which has high poverty rates and many racial minorities.
My focus when working with the students was to foster an excitement for computing that would stay with them after the program finished.
I continue to meet regularly with the teacher who led the program about ways he can incorporate computing into the standard curriculum using tools like Scratch. 
%I hoped to get the young students excited about future careers in coding that could help lift them from poverty.
As a university professor, 
I will help my university students find similar opportunities to help the disadvantaged members of their local communities.

\vspace{-0.15in}
\subsection*{Globally.}

I believe computer science education can build bridges between very different cultures.
To this end, I went to North Korea to teach at the Pyongyang University of Science and Technology (PUST).
PUST is a cooperative effort between the North and South Korean departments of education, the US State Department, and several non-governmental organizations.
Its mission is to contribute to the peaceful reunification of North and South Korea by internationalizing the North Korean educational system.
The students who attend PUST are isolated from the global economy,
and as a result they are extremely poor.
PUST teaches these students the skills needed to integrate into the global economy,
and the hope is that this will help lift these student out of poverty.
My work at PUST involved teaching these students how to collaborate with other students internationally via open source software,
and this work earned awards from both the California Assembly and US Congress.
As a professor at an American university,
I hope to participate in exchange programs with students and develop collaborations with universities from around the world.
%I believe open source software is a great tool to make this happen.
%The main tool I plan to use for this is open source software,

%- Bring this home by engaging with 
%On Campus
%- Research topics of robust estimation and 
%- Open source software and global collaborations.
%- Mentored hispanic undergraduates

\vspace{-0.15in}
\subsection*{For Veterans.}

I mentor veteran and active duty soldiers to help them understand their legal rights.
In particular, I mostly work with soldiers who are trying to be recognized as conscientious objectors.
These soldiers believe that their religion and military obligations are incompatible,
and I help them understand their legal rights.
I believe that people of all faiths should have equal access to these legal protections,
and I have had the opportunity to mentor Christians, Buddhists, Muslims, and Agnostics.
Besides direct mentoring, I also do educational and advocacy work.
I've worked with the ACLU and GI Rights hotline;
I've given a number of community presentations at churches and a TedX talk;
and I've written a number of online articles.
The organization Veterans for Peace gave me a ``Peace Day Award'' for these efforts.
%While this work has not directly involved computer science education,
%I believe the experience of working with veterans and religious minorities in the classroom.
Many American students are veterans,
and as a professor I will continue to help these students understand their rights and help them transition from the military into civilian society.

%Experience as a conscientious objector.
 %- Continue to work for and help veterans.
 %- Veterans for Peace.
 %- GI Rights Hotline.
 %- TedX talk, many presentations.
 %- Outreach in the form of writing blog posts, telling my own story ...
 %- Work with many faith traditions, and help others through the conscientious objection process from many faiths.

\end{document}


