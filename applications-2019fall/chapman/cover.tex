\documentclass[12pt]{article}

\usepackage[margin=0.95in]{geometry}
\usepackage{url}
\usepackage{tabularx}
\usepackage{xcolor}

\newcommand{\school}[2]{\textcolor{red}{\textbf{{#1}:} {#2}}}

\begin{document}

\noindent
\hspace{-0.13in}
\begin{tabularx}{1.03\textwidth}{Xr}
 & Mike Izbicki \\
 & 206 Knox Ct\\
 & Riverside, CA 92507\\
 & mike@izbicki.me\\
\end{tabularx}

\vspace{0.25in}

\setlength{\parskip}{15pt plus4mm minus3mm}
\pagenumbering{gobble}

\vspace{-0.4in}
\noindent
Dear faculty search committee:

\noindent
%My name is Mike Izbicki,
I am applying for the tenure-track Professor of Computer Science position advertised on your website.
I'm currently a postdoc at UC Riverside's Data Science Center,
where my research applies machine learning to large scale social media data.
I also have prior teaching experience at universities in both the United States and Korea, and
industry experience from 7 years in the US Navy working on cyber-security and nuclear reactors.
I believe this experience makes me a great fit for your department.

\noindent
My research directly supports Chapman's vision to ``prepare students to contribute to a global society''. 
In particular, social media platforms are generating more data everyday from around the world,
and I study new large scale learning algorithms designed to better understand this data.
For example, I've developed techniques for learning from the more than 100 non-English languages used daily on Twitter.
I've also developed new theoretical techniques for distributed machine learning using large clusters of computers and methods that are robust to maliciously corrupted training data.

\noindent
I enjoy teaching,
and I'm able to teach a broad range of courses at both graduate and undergraduate levels.
I've previously taught 11 university courses,
which ranged from introductory courses for humanities majors to advanced graduate-level machine learning courses.
Students like my teaching and have consistently rated my ``effectiveness as a teacher overall'' in at least the 80th percentile.
At Chapman, I'm particularly interested in teaching your Artificial Intelligence (CPSC 390), Data Science (CPSC 392), Machine Learning (CPSC 393), Algorithm Analysis (CPSC 406), and High Performance Computing (CPSC 445) courses,
but I'm also able to teach any course in the undergraduate curriculum when needed.
At the graduate level, I'm particularly interested in teaching Data Mining (CS 520) and Machine Learning (CS 613),
and in developing new courses on deep learning, numerical optimization, and parallel/distributed computing.
I'm committed to making my classrooms an inclusive environment where all students can succeed,
and I have experience working with students from many cultures both in the United States and Korea.

\noindent
One of my goals as a professor is to get students more involved with open source software.
I designed some of the first courses to teach students how to contribute to open source software,
and the resulting curricula earned an award from SIGCSE's graduate student research competition.
Through these courses and other mentoring, 
I've helped students contribute to and start their own open source projects with collaborators from around the world.
At Chapman, I hope to offer a course on open source software development,
and to integrate open source more closely with the students' other courses.

\noindent
On a more personal note, I'm particularly excited about teaching at Chapman because I have family who attended Chapman's film and business schools. 
%Thank you for your consideration.
%I look forward to hearing from you soon.
If you have any questions,
please do not hesitate to ask.

\noindent
Regards,

\vspace{-0.05in}
\noindent
Mike Izbicki

%\school{?}{ Christopher T. Ryu: machine learning; Paul Salvador Inventado: data mining for education; }
%
%\school{cpp}{Hao Ji: machine learning and hpc; Amar Raheja: image analysis; Daisy Tang: machine learning for robots; Lan Yang: big data; }
%
%\school{csusb}{Haiyan Qiao: machine learning applied to bioinformatics and games; George M. Georgiou: neural networks with complex numbers;  Qingquan Sun: Compressive Sensing and Machine Learning; Yan Zhang: data mining, distributed computing; }
%
%\school{HMC}{Ben Wiedermann: programming languages; Christopher A. Stone: type theory; George D. Montañez: why machine learning works (hired in 2017); Julie Medero: NLP}
%
%\school{CMC}{Dr. Jeho Park is a co-founder and the Director of Data Analysis for SoDAVi (Social Data Analysis and Visualization), a non-profit organization. He was also elected President (2019 term) of the Korean-American Computer Scientists and Engineers Association. His research and career interests are in high performance computing, data science, AI/Machine Learning, and computer science and mathematics education.}
%
%\school{SDSU}{Dr. Tao Xie, Dr. Mary Thomas: high performance and parallel computing; Dr. Marko Vuskovic: machine learning for robots; Dr. Faramarz Valafar: machine learning for bioinformatics; Dr. Marie Roch: machine learning for mapping marine biology}


\end{document}
