\documentclass[12pt]{article}

\usepackage[margin=1in]{geometry}
\usepackage{url}
\usepackage{tabularx}
\usepackage{xcolor}

\newcommand{\school}[2]{\textcolor{red}{\textbf{{#1}:} {#2}}}

\begin{document}

\noindent
\hspace{-0.13in}
\begin{tabularx}{1.03\textwidth}{Xr}
 & Mike Izbicki \\
 & 206 Knox Ct\\
 & Riverside, CA 92507\\
 & mike@izbicki.me\\
\end{tabularx}

\vspace{0.2in}

\setlength{\parskip}{15pt plus4mm minus3mm}
\pagenumbering{gobble}

\noindent
Dear faculty search committee:

\noindent
I am applying for the Assistant Professor of Computer Science position advertised on \mbox{\url{cra.org}}.
I'm currently a postdoc at UC Riverside's Data Science Center,
where my research applies machine learning to big social media data.
I have experience teaching diverse student populations both in the United States and Korea,
%Prior to that I held teaching appointments in both the United States and Korea, 
%where I taught both undergraduate and graduate students.
and I also have industry experience from 7 years in the US Navy working on cyber-security and nuclear reactors.
I believe this experience makes me a great fit for your department.

\noindent
My research directly supports Harvey Mudd's strategic visions of diversity and global engagement.
In particular, social media platforms are generating more data everyday from around the world,
and I study new large scale learning algorithms designed to better understand this data.
For example, I've developed techniques for understanding the more than 100 non-English languages used everyday on Twitter, and 
techniques to prevent malicious users from corrupting machine learning algorithms with racist messages or fake news.
%methods for robust machine learning that prevent malicious users from teaching 
I'm excited to help students find research projects that use the latest machine learning techniques to improve our world.

%\noindent
%My research develops theoretical methods that make large scale machine learning easier to use.
%In the distributed setting, I focus on algorithms that are easy to implement but still have strong theoretical guarantees.
%For example, I developed the first MapReduce algorithm for convex machine learning algorithms with statistically optimal generalization performance.
%I also have a number of Haskell libraries that develop new programming language features specifically designed for machine learning.
%For example, the \texttt{HerbiePlugin} automatically converts source code into numerically stable forms,
%so programmers don't have to worry about the intricacies of floating point arithmetic.

\noindent
I enjoy teaching,
and I'm able to teach a broad range of courses.
I've previously taught 11 university courses,
which ranged from introductory courses for humanities majors to advanced graduate-level machine learning courses.
Students like my teaching and have consistently rated my ``effectiveness as a teacher overall'' in at least the 80th percentile.
I'm particularly interested in teaching courses on algorithms, numerical optimization, machine learning, and programming languages,
but I'm also able to teach any course in the undergraduate curriculum.
I have experience working with students from many cultures in the US and Korea,
and I'm committed to making my classrooms an inclusive environment where all students can succeed.

\noindent
One of my goals as a professor is to help students contribute to open source software.
%Open source is particularly important to the mathematical community due to the many projects like R, SciPy, Sage, and Octave that make it easy for anyone in the world to make advanced mathematical models.
I have previously designed some of the first courses to teach students how to contribute to open source projects,
and the resulting curricula earned an award from SIGCSE's graduate student research competition.
Through these courses and other mentoring, 
I've helped students contribute to and start their own open source projects with collaborators from around the world.
I hope at Harvey Mudd to offer a course on open source software development
and to integrate open source more closely with the students' other courses.

\noindent
Thank you for your consideration.
If you have any questions,
please do not hesitate to ask.

\noindent
Regards,

\vspace{-0.10in}
\noindent
Mike Izbicki

%\school{?}{ Christopher T. Ryu: machine learning; Paul Salvador Inventado: data mining for education; }
%
%\school{cpp}{Hao Ji: machine learning and hpc; Amar Raheja: image analysis; Daisy Tang: machine learning for robots; Lan Yang: big data; }
%
%\school{csusb}{Haiyan Qiao: machine learning applied to bioinformatics and games; George M. Georgiou: neural networks with complex numbers;  Qingquan Sun: Compressive Sensing and Machine Learning; Yan Zhang: data mining, distributed computing; }
%
%\school{HMC}{Ben Wiedermann: programming languages; Christopher A. Stone: type theory; George D. Montañez: why machine learning works (hired in 2017); Julie Medero: NLP}
%
%\school{CMC}{Dr. Jeho Park is a co-founder and the Director of Data Analysis for SoDAVi (Social Data Analysis and Visualization), a non-profit organization. He was also elected President (2019 term) of the Korean-American Computer Scientists and Engineers Association. His research and career interests are in high performance computing, data science, AI/Machine Learning, and computer science and mathematics education.}
%
%\school{SDSU}{Dr. Tao Xie, Dr. Mary Thomas: high performance and parallel computing; Dr. Marko Vuskovic: machine learning for robots; Dr. Faramarz Valafar: machine learning for bioinformatics; Dr. Marie Roch: machine learning for mapping marine biology}


\end{document}
