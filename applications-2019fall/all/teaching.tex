\documentclass[12pt]{article}

\usepackage[margin=1in]{geometry}
\usepackage{url}
%\usepackage{hyperref}

\usepackage{titlesec}
%\titleformat{\section}[runin]
    %{\normalfont\bfseries}{\thesection}{1em}{}
\titleformat{\subsection}[runin]
    {\normalfont\bfseries}{\thesubsection}{0em}{\vspace{-0.00in}}

%\setlength{\parskip}{15pt plus4mm minus3mm}

\newcommand{\newpar}{\vspace{0.05in}\noindent}
\newcommand{\ignore}[1]{}

\begin{document}


\section*{Teaching Statement for Mike Izbicki}

\subsection*{Executive Summary.}
I enjoy teaching and have taught eleven courses at universities in both the United States and Korea.
I'm proud to say that my teaching has been recognized by students, researchers, and even congressmen.
Most of my student end-of-course evaluations ranked my ``effectiveness as a teacher overall'' in the 80th percentile or higher;
I earned a SIGCSE graduate research award for innovative curriculum design;
and I earned a Congressional Recognition award for my teaching in Korea.

A unique aspect of my teaching is an emphasis on helping students contribute to open source software.
For example, I developed some of the first curricula specifically designed to teach students how to start and contribute to open source projects.
I plan to incorporate open source practices into all the courses I teach,
and hope to create a center dedicated to teaching open source practices to students.


\vspace{-0.15in}
\subsection*{Teaching at UC Riverside.}
 had many teaching opportunities while working on my PhD.
After being a teaching assistant for a year, 
the department asked me to teach my own courses,
and I taught a total of 7 courses at all levels of the undergraduate curriculum.
These courses ranged from introductory courses for humanities majors to upper division algorithms courses.
%I taught courses for non-majors (CS006: Intro to the World Wide Web),
%lower level major courses (CS014: Data Structures, and CS100: Software Construction),
%and upper level major courses (CS141: Algorithm Design).
I took student end-of-course feedback seriously and actively worked to improve my teaching.
As a result, student evaluations of my ``effectiveness as a teacher overall'' steadily improved.
In my first class, I was ranked in the 46th percentile; 
then in subsequent courses I was ranked in the 62nd, 69th, 100th, 87th, 80th, and 93rd percentiles respectively.
Outside the classroom,
I coached UCR's International Collegiate Programming Contest (ICPC) team,
mentored undergraduate student projects,
and helped run a summer programming workshop for elementary and middle school students.

I'm particularly proud of the software construction course that I taught.
I redesigned the course's curriculum to focus on version control systems and open source software development methodologies,
which were not previously taught in UCR's curriculum.
I chose to emphasize these skills because research has shown that employers prefer to hire new graduates with this knowledge.
%My mantra for the course was ``everything in the course should be open sourced.''
At the time, very few courses existed to teach about open source software,
and so I designed the curriculum from scratch.
One of the most important teaching techniques I employed in the course was to put all the course material into a github repository,%
\footnote{The course details are all still available on github at \url{https://github.com/mikeizbicki/ucr-cs100}}
including the syllabus, assignments, textbook, sample programs, and grading scripts.
If a student found a mistake,
they could submit a pull request to fix it and get extra credit.
Students loved this system, and submitted an average of 5 pull requests per student.
%I enjoyed watching the students find and fix these mistakes,
%and the students did too.
%On average 5 pull requests were merged per student.
I found that shy students who participated only a small amount in labs and lecture felt comfortable contributing to the course in this format.
I believe that the best courses are created when educators work together and share their ideas,
so I submitted this course to SIGCSE2016's graduate student research competition,
where it won third place.
I also published a paper on the class at a Korean engineering conference called ICOPUST.

Mentoring student projects is another aspect of teaching I particularly enjoy. 
As a mentor, my main goal is to help students create open source projects that are both fun for them to build and useful to other people.
A common theme I've noticed is that while students are great at the technical aspects of building a project,
they often struggle with the softer skills needed to market their projects and get them widely adopted.
I like working closely with students to advertise their projects in appropriate social media forums like Reddit and Hacker News.
As a result, their projects have garnered thousands of github stars and active users.
This popularity makes the students excited and gives them a much stronger resume when looking for their first job.
Some example projects I've mentored are: 
the Git Game\footnote{\url{https://github.com/git-game}} teaches the player how to use git commands;
PacVim\footnote{\url{https://github.com/jmoon018/PacVim}} is a Pacman clone that teaches the player how to use vim commands;
and the Melody Matcher\footnote{\url{https://github.com/MiaoXiao/Melody-Matcher}} is an in-browser game designed for musical ``ear training.''

\vspace{-0.15in}
\subsection*{Teaching in Korea.}
I spent two semesters as a visiting professor at a unique university called the Pyongyang University of Science and Technology (PUST). 
PUST is located in North Korea, 
and was founded in a cooperative effort between the North and South Korean departments of education, the US State Department, and several non-governmental organizations.
Its mission is to contribute to the peaceful reunification of North and South Korea by internationalizing the North Korean educational system.
The students who attend PUST are isolated from the global economy,
and as a result they are extremely poor.
PUST teaches these students the skills needed to integrate into the global economy,
and the hope is that this will help lift these student out of poverty.
My work at PUST earned awards from both the California Assembly and US Congress.
%My experience teaching at PUST has made me a better educator because I am able to adapt to very different cultural situations,
%and I can help students understand the potential broader impacts of their computer science education.

A key part of my work at PUST was teaching North Koreans to use and contribute to open source software.
%This required navigating a number of political hurdles and clearly articulating to North Korean stakeholders of the benefits of open source software to their population.
%I taught four courses at PUST (Discrete Math, two sections of Algorithm Design, and Open Source Machine Learning Software). 
%I taught three courses at PUST.
%One of my main goals at PUST 
%At PUST I taught three courses:
%discrete math,
%algorithm design (I was the first non-North Korean to teach this course),
%and open source machine learning (I designed this course to meet the particular needs of PUSTs students).
%I was the first non-North Korean to teach the Algorithm Design course at PUST.
For example, I designed a masters-level open source machine learning course,
which was the first course in North Korea on open source software. 
I am particularly proud of these students' accomplishments.
Students in this course created patches for two widely used open source machine learning projects (Vowpal Wabbit and MLPack) which have been incorporated into the projects,
and I believe these patches are the first open source contributions created by North Koreans.
I'm amazed that because of open source software, these students have had the opportunity to collaborate with people from very different cultures around the world.

%I see my experience at PUST making me a better educator in the US.
%I can help them understand how different countries have different computational needs,
%and how computer science can play a role in economic development and American policy making.

\vspace{-0.15in}
\subsection*{Future Teaching Goals.}

I am most interested in teaching classes on data structures, discrete math, algorithms, theory of computing, artificial intelligence, data science, machine learning, deep learning, numerical optimization and programming languages.
I am also able to teach any course in the undergraduate computer science curriculum if needed.

A longterm goal of mine is to create a center for teaching open source software development.
A first step to achieving this goal would be to add a course on open source software development to the curriculum.
This course would get students directly involved in contributing to widely used open source projects and teach them how to start their own open source projects.
Once a number of students are trained in open source,
the center could then provide longterm support to open source projects that students have created.
%This would also be a good opportunity for interdisciplinary collaboration,
Ultimately, the center would encourage collaboration with developers from around the world and increase the profile of the host university.
%Students would learn how to collaborate with developers from around the world,
%and would contribute to products.

\end{document}

