\documentclass[12pt]{article}

\usepackage[margin=1in]{geometry}
\usepackage{url}
\usepackage{tabularx}
\usepackage{xcolor}

\newcommand{\school}[2]{\textcolor{red}{\textbf{{#1}:} {#2}}}

\begin{document}

\noindent
\hspace{-0.13in}
\begin{tabularx}{1.03\textwidth}{Xr}
 & Mike Izbicki \\
 & 206 Knox Ct\\
 & Riverside, CA 92507\\
 & mike@izbicki.me\\
\end{tabularx}

\vspace{0.25in}

\setlength{\parskip}{15pt plus4mm minus3mm}
\pagenumbering{gobble}

\noindent
Dear faculty search committee:

\noindent
I am applying for the tenure-track Professor of Computer Science position advertised on \url{cra.org}.
I'm currently a postdoc at UC Riverside's Data Science Center,
where my research applies deep learning to large scale social media data.
I also have prior teaching experience at universities in both the United States and Korea, and
industry experience from 7 years in the US Navy working on cyber-security and nuclear reactors.
I believe this experience makes me a great fit for your department.

\noindent
My research develops new theoretical and practical deep learning methods.
%In particular, social media platforms are generating more data everyday from around the world,
%and I study new large scale learning algorithms designed to better understand this data.
On the practical side, I've developed a technique called the Unicode convolutional neural network (UnicodeCNN) for understanding highly multilingual text corpora,
and I applied the UnicodeCNN to analyzing a dataset of 1 billion tweets written in more than 100 languages from around the world.
On the theoretical side, I've developed methods to make machine learning algorithms robust to outliers in the training dataset, 
and algorithms for distributed machine learning with strong theoretical guarantees.

\noindent
I enjoy teaching,
and I'm able to teach a broad range of courses at both graduate and undergraduate levels.
I've previously taught 11 university courses,
which ranged from introductory courses for humanities majors to advanced graduate-level machine learning courses.
Students like my teaching and have consistently rated my ``effectiveness as a teacher overall'' in at least the 80th percentile.
I'm particularly interested in teaching undergraduate and graduate level courses on algorithms, machine learning, artificial intelligence, and data science,
but I'm also able to teach any course in the undergraduate curriculum.
I'm committed to making my classrooms an inclusive environment where all students can succeed.

\noindent
One of my goals as a professor is to get students more involved with open source software.
I designed some of the first courses to teach students how to contribute to open source software,
and the resulting curriculum earned an award from SIGCSE's graduate student research competition.
Through these courses and other mentoring, 
I've helped students contribute to and start their own open source projects.
I specifically hope to offer a course on open source software development,
and to integrate open source more closely with the students' other courses.

\noindent
Thank you for your consideration.
If you have any questions,
please do not hesitate to ask.

\noindent
Regards,

\vspace{-0.15in}
\noindent
Mike Izbicki

%\school{?}{ Christopher T. Ryu: machine learning; Paul Salvador Inventado: data mining for education; }
%
%\school{cpp}{Hao Ji: machine learning and hpc; Amar Raheja: image analysis; Daisy Tang: machine learning for robots; Lan Yang: big data; }
%
%\school{csusb}{Haiyan Qiao: machine learning applied to bioinformatics and games; George M. Georgiou: neural networks with complex numbers;  Qingquan Sun: Compressive Sensing and Machine Learning; Yan Zhang: data mining, distributed computing; }
%
%\school{HMC}{Ben Wiedermann: programming languages; Christopher A. Stone: type theory; George D. Montañez: why machine learning works (hired in 2017); Julie Medero: NLP}
%
%\school{CMC}{Dr. Jeho Park is a co-founder and the Director of Data Analysis for SoDAVi (Social Data Analysis and Visualization), a non-profit organization. He was also elected President (2019 term) of the Korean-American Computer Scientists and Engineers Association. His research and career interests are in high performance computing, data science, AI/Machine Learning, and computer science and mathematics education.}
%
%\school{SDSU}{Dr. Tao Xie, Dr. Mary Thomas: high performance and parallel computing; Dr. Marko Vuskovic: machine learning for robots; Dr. Faramarz Valafar: machine learning for bioinformatics; Dr. Marie Roch: machine learning for mapping marine biology}


\end{document}
